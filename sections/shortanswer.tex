\documentclass[../main.tex]{subfiles}
\begin{document}
\section*{Short Answer (Open Ended)}
    \begin{questions}
    \setcounter{question}{25}
   
    % Units 1 & 2
    \question[1] Use this table of frequencies for colors of Skittles in a bag to solve the following problems.
    \begin{center}
            \begin{tabular}{ |c|c|c|c|c|c| } 
            \hline
            red & orange & yellow & green & blue & brown \\
            \hline
            0.1 & 0.2 & 0.1 & 0.2 & 0.1 & 0.3 \\
            \hline
            \end{tabular} 
        \end{center}
        
        \begin{parts}
            \part What is the probability of selecting one orange or green Skittle out of the bag? \vspace{\stretch{1}}
    
            \part What is the probability of \textbf{not} selecting one yellow or one blue Skittle out of the bag? \vspace{\stretch{1}}
    
            \part Suppose there are two bags of Skittles. What is the probability of picking one brown from one bag and one red from the other bag?  \vspace{\stretch{1}} \end{parts}
    
    \question[1] You are rolling two dice and are taking the sum of the numbers rolled. 
        \begin{parts}
            \part State the probability of rolling a 6 or 11. \vspace{\stretch{1}}
    
            \part State the probability of rolling a number divisible by 2. \vspace{\stretch{1}}
            
            \part State the probability of rolling a double-digit number. \vspace{\stretch{1}} \end{parts} 
    \newpage
    \question[1] Given $129^\circ$, express one positive coterminal angle and one negative coterminal angle.
    \vspace{\stretch{1}}
    
    \question[1] From among 8 projects for consideration, the mayor must put together a prioritized list of 3 projects to submit to the city council for funding. How many lists can be formed? 
    \vspace{\stretch{1}}
    
    \question[1] Convert $\frac{3\pi}{2}$ from radians to degrees \textbf{AND} state which quadrant the angle is in. The answer should be simplified to the tenth. \vspace{\stretch{1}} 
    
    % Unit 3
    \question[1] Determine the domain of $f(x) = \frac{5}{x^2 - 4}$ algebraically. \vspace{\stretch{1}}
    
    \question[1] Determine the domain of $f(x) = \sqrt{x^2 + 13} $ algebraically. \vspace{\stretch{1}}
    
    \question[1] Given $f(x) = 3x - 5$ and $g(x) = x - 2$, find $f(g(x))$. \vspace{\stretch{1}}
    
    \question[1] Given $f(x) = x^2 + 1$ and $g(x) = \frac{1}{x^2 + 4}$, find $f(g(x))$. \vspace{\stretch{1}}
         
    % Unit 4
    \question[1] Determine the slant asymptote of $f(x) = \frac{x^2 - 6x + 7}{x + 5}$ algebraically. \vspace{\stretch{1}}
    
    \question[1] Determine the slant asymptote of $f(x) = \frac{3x^3 - 2x}{-2x^2 + 4}$ algebraically. \vspace{\stretch{1}}
    
    \newpage
    \question[1] Solve the equation. $\frac{4x}{x - 1} + \frac{6}{x + 2} = \frac{24}{x^2 + x - 2}$
    \vspace{\stretch{1}}
    
    % Unit 5
    \question[1] Expand the following (rewrite as an expanded logarithm): $\ln(7xy)$. \vspace{\stretch{1}}
    
    \question[1] Condense the following (rewrite as a single, condensed logarithm): $\log 12 + \log x + \log y$. \vspace{\stretch{1}}
    
    \question[1] The number B of a flesh-eating bacteria in a petri dish culture after t hours is given by $$B = 68e^{0.867t}$$
        \begin{parts}
            \part What is the initial amount of bacteria present? \vspace{\stretch{1}}
    
            \part What is the rate that the bacteria are increasing at? \vspace{\stretch{1}}
            
            \part How many bacteria are present after 8 $\frac{1}{2}$ hours? \vspace{\stretch{1}} 
            
            \part How many bacteria are present after 3 days? \vspace{\stretch{1}} \end{parts}
     
    % Unit 6
    \question[1] Find the exact value of $\sin^{-1} (\frac{\sqrt{2}}{2})$ using the unit circle. \vspace{\stretch{1}}
    
    \question[1] Find the exact value of $\cos^{-1} (\frac{1}{2})$ using the unit circle. \vspace{\stretch{1}}
    
    \question[1] Using the unit circle, find the exact value of $\cos(150^\circ)$. \vspace{\stretch{1}}
    
    \question[1] Using the unit circle, find the exact value of $\csc(60^\circ)$. \vspace{\stretch{1}}
    
    \newpage
    \question[1] State the amplitude, period, phase shift and vertical shift of $$y = 7\cos(x - \pi) + 6$$. 
    \newline
    Amplitude: \line(1,0){40} \\
    \newline
    Period: \line(1,0){40} \\
    \newline
    Phase Shift: \line(1,0){40} \\
    \newline
    Vertical Shift: \line(1,0){40}
    \vspace{\stretch{0.5}}
 
    % Unit 7
    \question[1] Simplify $\cot\theta\csc\theta$. \vspace{\stretch{1}}
    
    \question[1] Verify that $\cos x + \sin x + \tan x$ is $\sec x$ \vspace{\stretch{1}}
    % This problem was taken directly from a past test.
    
    \question[1] Verify that $\frac{1}{1 - \sin\theta} + \frac{1}{1 + \sin\theta}$ is $2\sec^2\theta$ \vspace{\stretch{1}}
    % This problem was taken directly from a past test.
    
    \question[1] Find all values of $x$ in the interval $[0, 2\pi)$ that solves $2\cos x + 2 = 0$. The answer can be provided in either radians or degrees. \vspace{\stretch{1}}
    
    \newpage
    \question[1] Solve $\triangle ABC$ given that $\angle A = 60^\circ, b = 15$ and $c = 22$. \vspace{\stretch{2}}
    
    \question[1] Solve $\triangle ABC$ given that $a = 20, b = 30$ and $\angle B = 140^\circ$. \vspace{\stretch{2}}
    
    % Unit 8
    \question[1] Solve the non-linear system: $$x^2 + y = 6$$
    $$x + y = 14$$ \vspace{\stretch{1}}
    
    \question[1] Convert $y = 2x + 1$ from rectangular to polar form. \vspace{\stretch{1}}
    
    \question[1] Convert $r(2\cos\theta + \sin\theta) = 4$ from polar to rectangular form. \vspace{\stretch{1}}
    
    \question[1] Perform the following matrix operation:
    \begin{bmatrix} 
    1 & -7 \\
    0 & 3 \\
    5 & -10
    \end{bmatrix}
    $+$
    \begin{bmatrix} 
    11 & 17 \\
    -12 & 8 \\
    4 & 9
    \end{bmatrix}
    \vspace{\stretch{1}}
    \end{questions}
\end{document}